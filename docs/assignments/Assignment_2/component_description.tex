After identifying the main subsystems being the Robot, Control station and the Interface, the components responsible for services and functionality were created for each subsystem. They are described for each subsystem below.
\subsection*{Subsystem 1: Robot}
The Robot subsystem runs on the rover and consists of sensors, actuators, a communications device and a controller.
\subsubsection*{Sensor components}
The sensor components acts as adapters between the \emph{hardware/simulator} and the controller by providing  an implementation of the appropriate interface; location, proximity or camera.
\subsubsection*{Actuator components}
Acs as an adapter between the \emph{hardware/simulated} actuators and the controller by providing an interface for the controller to use for controlling the robot.

\subsubsection*{Communicator}
This component provides an abstraction for communication with the control station.

\subsubsection*{Controller}
The main system/program of the robot. The controller component requires sensor interfaces for input, communication interfaces for talking thith the controll station and actuator interfaces for doing stuff.

\subsection*{Subsystem 2: Control station}
The control station orchestrates the communication between the operators and robots by offering interfaces to the other subsystems.
\subsubsection*{Sender}
The sender component is responsible for sending instructions to to the rovers. It sends information by implementing an interfaced used by other components to relay information to the rovers.
\subsubsection*{Receiver}
The sender component is responsible for receiving data from the rovers. the data is the sent to the storage interface for further processing.
\subsubsection*{Conductor}
The conductors is responsible for calculating a specific instruction for the robot, given a mission, reward procedure and strategy. This component realises the assigned mission interface to get new missions and make use of the dispatch interface to send instructions.
\subsubsection*{Point Rewarder}
Calculates points given the information from the robot status interface. Provides the number of points and the current reward procedure through realisation of interfaces.
\subsubsection*{Robot Tracker}
Keeps track of all robots and provide data about the robots through the robot status interface. Realises the store interface to retrieve the data.
\subsubsection*{Info provider}
Provides information to an interface such as points rewarded and the location of the robots in the environment by using the points and robot status interfaces of the point rewarder and robot tracker.

\subsection*{Subsystem 3: Interface}
The interface subsystem provides the relevant information about the environment to operators and allows them to design missions and send commands. Since the design of the interface may vary, this subsystem contains adapter components which may be used for implementing a specific interface.
\subsubsection*{Display}
The display provides the information to display in order to provide a visualisation of reward points and robot locations in the environment. It listens to the state of the environment sent by the control station and keeps the last known state for display.
\subsubsection*{Action button}
This component lets an interface instance map a trigger to an action for the control station. Actions, such as \textit{emergency stop} are relayed directly to the concerned robot by the \textit{Sender} component.
\subsubsection*{Mission Composer}
The mission composer is a way for the interface to assign a composed mission to a robot by letting the operator choose the relevant points and strategy for the mission.